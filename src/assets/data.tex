\documentclass{article}

\title{IMAGISTICA PRIN REZONANȚĂ MAGNETICĂ NUCLEARĂ}
\date{}
\author{}

\begin{document}

    \maketitle

    \section{APLICAȚII ÎN STUDIUL SCLEROZEI MULTIPLE}
    
    Imagistica medicală a înregistrat un progres real odată cu introducerea RMN-ului ca tehnică imagistică de diagnostic.

    Principalele caracteristice ale acestei tehnici sunt neinvazivitatea și calitatea marea a imaginii. Scleroza multiplă, boală neurologicâ demielinizantă, este un exemplu relevant de aplicație a RMN-ului în studiul sistemului nervos central. 

    \section{GENEZA SEMNALULUI RMN}
        Metoda se bazează pe fenomenul fizic al rezonanței magnetice nucleare. 
        
        O imagine obținută traduce în semnale optice intensitatea semnalelor de radiofrecvență (RF) ale unor nuclei atomici din structuri anatomice examinate. 
        
        Metoda se bazeazâ pe proprietatea unor nuclei atomici, în special a celor de hidrogen (respectiv protonilor) de a realiza o mișcare de rotație în jurul propriului ax, adică de a avea un moment cinetic propriu, spinul nuclear.
        
        Rotația unei particule încărcate electric, cum este protonul, determinâ apariția unui câmp magnetic propriu orientat în sens opus câmpului electric.
        
        Această transformă fiecare nucleu într-un dipol magnetic, un magnet microscopic.

        \section{}
        <()>

        <()>
